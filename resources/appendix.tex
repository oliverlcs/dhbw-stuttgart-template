%!TEX root = ../main.tex

\addchap{Anhang}
\begin{itemize}
	\item \ref{apx:1} Anhang 1 \dotfill{} \pageref{apx:1}
	\item \ref{apx:2} Anhang 2 \dotfill{} \pageref{apx:2}
\end{itemize}

\chapter{Anhang 1}
\label{apx:1}

%!TEX root = ../main.tex

\chapter{Aufbau und Alterung von Fahrzeugbatterien}
\paragraph{Disclaimer} Die Zitate für das zweite Kapitel sind unvollständig und werden noch hinzugefügt. Dieses Kapitel kommt in den Anhang.
\\\\
Elektrofahrzeuge nutzen überwiegend \acs{LIB}, die sich in Größe, Bauform und der chemischen Zusammensetzung ihrer Komponenten unterscheiden. Eine zentrale Rolle für die Leistungsfähigkeit der Batterie spielt die Kathode. Die hierfür eingesetzten Materialen bestimmen maßgeblich Energiedichte, Lade- und Entladeleistung sowie Lebensdauer. Obwhol der Begriff \enquote{Zellchemie} streng genommen die Gesamtheit aller elektrochemisch relevanten Materialen umfasst, einschließlich Anode, Elektrolyt und Separator, wird in der Praxis häufig nur die Zusammensetzung der Kathode bezeichnet. Dabei kommen in Elektrofahrzeungen vor allem drei Kathodenmaterialen zum Einsatz: \ac{LFP}, \ac{NMC} und \ac{NCA}.\\
\acs{LFP} zeichnet sich durch eine lange Lebensdauer und hohe thermische Stabilität aus, weist jedoch im Vergleich zu den \acs{NMC}- und \acs{NCA}-Technologien eine geringere Energiedichte auf. Im Gegensatz dazu ermöglichen \acs{NMC} und \acs{NCA} eine höhere Energiedichte, gehen jedoch mit einer geringeren thermischen Stabilität einher. Darüber hinaus werden \acs{NMC}-Zellen anhand ihres Massenverhältnis von Nickel, Mangan und Cobalt klassifiziert, z.B. als \acs{NMC}111, \acs{NMC}622 oder \acs{NMC}811. Die Zahlen geben die prozentuale Zusammensetzung der Metalle an, beispielsweise enthält eine \acs{NMC}622-Zelle ca. 60\% Nickel, 20\% Mangan und 20\% Cobalt. Ein erhöhter Nickelanteil sorgt für eine bessere Energiedichte zulasten der Lebensdauer. Diese Zellchemien finden Einsatz in unterschiedlichen Bauformen, wie Pouch- oder Zylinderzellen.
Pouch-Zellen bieten eine hohe Energiedichte und lassen sich flexibel an unterschiedliche Bauformen anpassen. Während des Alterungsprozesses neigen sie jedoch zur Expansion, was die Batteriealterung beschleunigt \cite{articlePouchZellenAlterung}. Hingegen zeichnen sich Zylinderzellen durch eine lange Lebensdauer und eine robuste Bauform aus. Dafür ist die Kühlung bei Rundzellen aufwendiger und die Energiedichte bei gleichem Bauvolumen oftmals geringer.
\section{Hauptbestandteile von \acs{LIB} und ihre Funktionen}
Der elektrische Strom in einer \acs{LIB} fließt, wenn ein innerer Ionenfluss und ein äußerer Elektronenfluss besteht. Diese beiden Flüsse ermöglichen die verschiedenen Komponenten der \acs{LIB}. Sie besteht aus Kathode, Anode, Elektrolyt, Separator, Stromkollektoren und Bindemittel \cite{urlIdAufbauBatterie}, wie in Abbildung dargestellt. 


Die Kathode nimmt Elektronen auf (Reduktion) und lagert Lithium-Ionen (\ce{Li+}-Ionen) ein (Interkalation), während die Anode Elektronen abgibt (Oxidation) und \ce{Li+}-Ionen auslagert (Deinterkalation). Das Elektrolyt transportiert diese \ce{Li+}-Ionen zwischen Anode und Kathode. Der Separator lässt Ionen passieren, verhindert jedoch den Durchtritt von Elektronen. Dadurch unterbindet er den direkten Kontakt der Elektroden, der zu einem Kurzschluss führen würde. Damit ermöglichen Elektrolyt und Separator den inneren Ionenfluss zwischen Kathode und Anode. Die Stromkollektoren sorgen für eine leitfähige Verbindung zwischen Elektrodenmaterial und äußerem Stromkreis, sodass Elektronen effizient abgeleitet werden. Die Bindemittel fixieren Anode und Kathode an ihrem jeweiligen Stromkollektor und gewährleisten mechanische und chemische Stabilität.\\
Beim Entladeprozess deinterkaliert Lithium aus der Graphitstruktur der Anode und oxidiert zu einem \ce{Li+}-Ion unter Abgabe eines Elektrons. Dieses Elektron fließt über den Stromkollektor aus Kupfer zur Kathode, wo es Cobalt-Ionen (\ce{Co^{3+}}) in der \acs{NMC}-Kathode zu \ce{Co^{2+}} reduziert. Parallel dazu diffundiert das \ce{Li+}-Ion über das Elektrolyt und interkaliert sich in der Kristallstruktur der Kathode. Die zugrunde liegende Reaktion lässt sich vereinfacht wie folgt darstellen:
\begin{align*}
    \ce{Li &-> Li+ + e-}              &\quad& \text{(Oxidation beim Entladeprozess)} \\
    \ce{Co^{3+} + e- &-> Co^{2+}}     &\quad& \text{(Reduktion beim Entladeprozess)} \\
    \ce{Li + Co^{3+} &<=> Li^+ + Co^{2+}} &\quad& \text{(Redoxreaktion)}
\end{align*}

Beim Ladeprozess kehrt sich der Redoxprozess um und Anode und Kathode tauschen ihre räumliche Position. Das Cobalt an der Anode oxidiert und \ce{Li+}-Ionen deinterkalieren aus der Kristallstruktur. Die dabei abgegebenen Elektronen fließen vom Aluminium-Stromkollektor über den externen Stromkreis zum Kupfer-Stromkollektor zurück. Zeitgleich transportiert das Elektrolyt die \ce{Li+}-Ionen zur Graphitstruktur, wo die Interkalation stattfindet.\\
Die Alterungsmechanismen der Batterien stören diese zuvor erläuterten Prozesse, wodurch sich Batterieleistung und -kapazität verringern. Durch eine Analyse dieser Mechanismen lassen sich die wichtigsten Einflussfaktoren identifizieren. Insbesondere diese sollten im Datensatz für das Training der datengetriebenen Modelle berücksichtigt werden, da nicht-relevante Parameter Unschärfe in den Lernprozess einbringen. Dadurch wird eine genauere Vorhersage der Batterialterung ermöglicht. Aus diesem Grund beleuchtet das nächste Kapitel die Zusammenhänge zwischen Alterungsmechanismen und Einflussfaktoren.

\section{Alterung von Batterien}
Die Alterung von Batterien beschreibt den fortschreitenden Verlust von Leistung sowie Kapazität und lässt sich in kalendarische und zyklische Alterung unterteilen. Die Kalendarische Alterung erfolgt zeitabhängig und ist unabhängig von Ladezyklen, während zyklische Alterung primär durch Lade- und Entladeprozessen bedingt ist. Beide Alterungsformen lassen sich auf dieselben grundlegenden Mechanismen zurückführen: dem Verlust aktiven Lithiums, dem Abbau des aktiven Elektrodenmaterials, dem Rückgang des Elektrolyts und der Erhöhung des internen Widerstands.\\
Der Verlust des aktiven Lithiums (engl. \ac{LLI}) reduziert die Menge an verfügbaren \ce{Li+}-Ionen. Diese Ionen sind für den inneren Ionenstromkreis verantwortlich, der den Elektronenfluss im äußeren Stromkreis ermöglicht. Ohne ausreichend \ce{Li+}-Ionen können weniger Elektronen im äußeren Stromkreis fließen, was Leistung und Kapazität der Batterie limitiert.\\
Der Verlust des aktiven Materials von Kathode und Anode (engl. \ac{LAM}) verschlechtert die Inter- und Deinterkalation von \ce{Li+}-Ionen sowie die Oxidation und Redukiton von Elektronen. Das schränkt den Ionen- und Elektronenfluss ein, was Leistung und Kapazität begränzt.\\
Der Verlust des Elektrolyts (engl. \ac{LE}) verringert den Transport von \ce{Li+}-Ionen zwischen Kathode und Anode, was den Ionenfluss beeinträchtigt. Dies führt zu einem geringeren Elektronenfluss, wodurch sich Leistung und Kapazität minimieren. Häufig geht der Elektrolytabbau mit einem  Verlust von Lithium und aktiven Material einher.\\
Ein erhöhter interner Widerstand (engl. \ac{IR}) fördert die Wärmeentwicklung, wodurch die anderen Alterungsprozesse beschleunigt werden. Gleichzeitig führt er zu Spannungsabfällen, was die nutzbare Leistung reduziert.
Diese Mechanismen interagieren miteinander und werden durch interne sowie externe Prozesse beeinflusst. Eine Analyse der internen Prozesse ist erforderlich, um den Einfluss externen Faktoren auf die Batteriealterung zu verstehen. Letztere misst meist das \ac{BMS} und stellt damit die Datengrundlage für das Training der Modelle dar.

\subsection{Interne Faktoren}
Hinweis: Die folgenden Beschreibungen von Anode und Kathode beziehen sich ausschließlich auf den Entladeprozess, wie in Abbildung dargestellt.
\paragraph{\acs{SEI}} Bereits bei den ersten Ladezyklen bildet sich an der Grenzfläche zwischen Anode und Elektrolyt eine irreversible Grenzschicht, das sogennante \ac{SEI}. Im Zuge dieser Reaktion werden Lithium-Ionen irreversibel gebunden, wodurch sie dem elektrochemischen Prozess nicht mehr zur Verfügung stehen. Die \acs{SEI}-Schicht dient neben dem Separator als weitere ionenselektive Barriere, die nur Ionen, jedoch keine Elektronen passieren lässt. Mechanische Spannungen, wie die Expansion und Kontraktion der Graphitstruktur, führen zu Risse in der \acs{SEI}-Schicht. Sie regeneriert diese Instabilitäten unter Einbindung von \ce{Li+}-Ionen, wodurch sie verdickt. Dieser wiederholende Prozess führt zum \acs{LLI} und einem Anstieg des \acs{IR}. Das reduziert die nutzbare Kapazität und beschleunigt den Alterungsprozess. Damit erklärt die Bildung der \acs{SEI}-Schicht den anfänglichen Abfall der Batteriekapazität bei \acs{LIB}.

\paragraph{Lithium-Plating} Beim Lithium-Plating interkalieren \ce{Li+}-Ionen nicht in die Graphitstruktur der Anode, sondern lagern sich als metallisches Lithium ab. Dabei können sich Lithium-Dendriten bilden, die den Separator durchdringen können. Dies beschleunigt die Alterung der Batterie erheblich und kann im schlimmsten Fall zu internen Kurzschlüssen führen. Darüber hinaus steht das Lithium nicht mehr dem elektrochemischen Prozess zur Verfügung, das zusätzlich die Batteriekapazität reduziert.

\paragraph{Kathode und CEI} Die Kathode dient als Quelle für \ce{Li+}-Ionen, die beim Entladen interkaliert werden. Während zyklischer Lade- und Entladevorgänge unterliegt die Kathodenstruktur mechanischen Spannungen wie Expansion und Kontraktion, die zur Bildung von Mikrorissen, Partikelablösung (Peeling) und strukturellem Zerfall führen. Zusätzlich kann es zur Auflösung von Übergangsmetallionen wie \ce{Ni^{2+}}, \ce{Co^{2+}} und \ce{Mn^{2+}} kommen, die durch den Separator zur Anode diffundieren und dort die \acs{SEI}-Schicht destabilisieren. Dies trägt zur Kapazitätsreduktion (\acs{LLI}, \acs{LAM}) bei und erhöht den (\acs{IR}).\\
An der Kathodenoberfläche lagern sich bei den Redoxreaktionen Abbauprodukte des Elektrolyten ab. Diese führen zur Bildung eines stabilen festen Grenzfilms, des sogenannten \ac{CEI}. Das \acs{CEI} schützt zwar die Kathode vor weiterer Elektrolyt-Zersetzung und reduziert den irreversiblen Materialverlust, wird jedoch mit zunehmender Zyklenzahl dicker und trägt dadurch ebenfalls zu einem Anstieg des \acs{IR} bei.

\paragraph{Elektrolyt} Das Elektrolyt spielt eine zentrale Rolle im Transport von \ce{Li+}-Ionen zwischen den Elektroden. Während der Lade- und Entladezyklen kommt es zu Redoxreaktionen zwischen Elektrolyt und Elektroden, wodurch die Bildung von \acs{SEI} und \acs{CEI} auf der Anode bzw. Kathode initiiert wird. Mit der fortschreitenden Nutzung wird das Elektrolyt zunehmend konsumiert, was zu einem kontinuierlichen Abbau und Neubildung dieser Grenzschichten führt. Die sich verdickenden Filme erhöhen den \acs{IR} und tragen somit zur Degradation der Batterie bei.\\
Zusätzlich bilden sich während der Elektrolyt-Zersetzung Gase, was eine Volumenexpansion der Batterie zur Folge hat und den mechanischen Stress erhöht. Dadurch wird der Separator zunehmend beschädigt, was die Alterung der Batterie beschleunigt.

\paragraph{Separator} Der Separator dient als Barriere, die den Durchtritt von Elektronen verhindert und ausschließlich den Transport von Ionen ermöglicht. Er weist eine poröse Struktur auf, die unter normalen Bedingungen eine effiziente Ionenmigration sicherstellt. Wenn der Separator jedoch mit dem Elektrolyten reagiert, wird seine Integrität beeinträchtigt. Dazu können sich die Dekompositionsprodukte der Elektroden, wie Metall-Ionen, auf der Oberfläche ablagern und Poren verstopfen. Diese Prozesse schränken den Ionenfluss ein und limiteren die Batterieleistung. 

\subsection{Externe Faktoren}

Die wesentlichen externen Faktoren, die die Batteriealterung beschleunigen, umfassen den Ladezustand, die Tiefenentladung, die Lade-Entlade-Rate und die Temperatur.

\paragraph{\acs{SoC}} Mit dem Ladezustand (engl. \ac{SoC}) ist das Verhältnis zwischen dem aktuellen Ladezustand und der maximalen Kapazität einer Batterie gemeint. Zwischen \acs{SoC} und der Batteriespannung besteht eine enge korrelation, beispielsweise beeinflussen die \acs{SoC}-Grenzen die Cut-Off-Spannung einer Batterie.\\
Ein hoher Ladezustand führt zu einer Erhöhung der Spannung an der positiven Elektrode und einer Reduktion der Spannung an der negativen Elektrode. Dies kann die Stabilität der positiven Elektrode beeinträchtigen und zu deren Ablösung führen. Dies fördert Nebenreaktionen, wie die \acs{SEI}- und \acs{CEI}-Bildung, wodurch sich die Leistung minimiert.\\
Zudem beschleunigt ein hoher \acs{SoH} die Kalendarische Alterung, da sich bei hohem Ladezustand vermehrt Gase an der positiven Elektrode bilden. Diese Gase expandieren das Batterievolumen und führen so zu struktureller Degradation, mechanischer Ermüdung und Partikelfraktur. Diese Folgen treiben die Batteriealterung deutlich voran.
\\
Andererseits können wiederholte Lade- und Entladeprozesse bei niedrigem \acs{SoC} zur Kupferauflösung an der Anode führen, was wiederum den \acs{LLI} und den \acs{IR} erhöht.\\
Das Batterie-Management-System (BMS) misst kontinuierlich den \acs{SoC} und sorgt dafür, dass die Batterie innerhalb sicherer Grenzen betrieben wird. Im Vergleich zur Lagerung bei 100\% \acs{SoC} bildet sich bei einem \acs{SoC} von etwa 50\% eine stabilere \acs{SEI}-Schicht, was die Lebensdauer der Batterie verlängert.

\paragraph{DOD} Mit der Tiefenentladung (engl. \ac{DOD})  ist der Anteil der Kapazität gemeint, der im Vergleich zur vollen Kapazität der Batterie entladen wurde. Ein tieferer \acs{DOD} fördert die Freisetzung von \ce{Li+}-Ionen und beschleunigt die Zersetzung des Elektrolyten. Zudem führt ein \acs{DOD} zu einer Volumenexpansion der Graphitanode während des Lade- und Entladezyklus, wodurch sich der Abstand zwischen den Graphitschichten vergrößert. Dies führt dazu, dass das schichtgelagerte Graphit freigelegt wird und mehr Oberfläche dem Elektrolyten ausgesetzt wird, was \acs{LLI} begünstigt.\\
Außerdem verformt sich die poröse Struktur des Separators bei einer Tiefenentladung von über 100\% erheblich. Dies ist auf die Zersetzung des Anodenmaterials und die Ablagerung dieser Zersetzungsprodukte auf der Separatoroberfläche zurückzuführen, wodurch die Poren blockiert werden. Infolgedessen ist die Diffusion von \ce{Li+}-Ionen während der Zyklusprozesse stark eingeschränkt, was zu einem erhöhten \acs{IR} führt und die Batteriealterung beschleunigt.

\paragraph{C-Rate} Die Lade- und Entladerate, auch als C-Rate bezeichnet, beschreibt den auf die Nennkapazität des Akkus bezogenen Lade- oder Entladestrom in Amperestunden. Eine hohe C-Rate bedeutet, dass der Akku intensiver belastet wird, was zu einer höheren Migration von \ce{Li+}-Ionen führt und die Zersetzung des Elektrolyten beschleunigt. Diese Zersetzung führt zu strukturellen Schäden an den Anoden- und Kathodenmaterialien. Zudem verursachen höhe Ströme eine lokale Überhitzung der Batteriematerialen, wodurch die Struktur deformiert und die Interkalation bzw. Deinterkalation von \ce{Li+}-Ionen beeinträchtigt wird. Als Folge können weniger \ce{Li+}-Ionen pro Zeiteinheit fließen und die Batterieleistung sinkt.

\paragraph{Temperatur} Der optimale Temperaturbereich für \acs{LIB} liegt zwischen 15 \textdegree C und 35 \textdegree C. Temperaturen oberhalb dieses Bereichs fördern den Abbau des Elektrolyten sowie die Bildung der \acs{CEI}. Zudem können Mikrorisse in der Kathode entstehen, was die elektronische Leitfähigkeit verringert und die Interkalation bzw. Deinterkalation von \ce{Li+}-Ionen behindert, was eine Minderung der Kapazität zur Folge hat.\\
Bei Temperaturen unterhalb des Idealbereichs sinkt die Ionenmobilität sowie die elektrochemische Aktivität des Elektrolyten. Zusätzlich kann es zu einer Reaktion zwischen dem Elektrolyten und metallischem Lithium an der Anode kommen, wodurch sich eine sekundäre \acs{SEI}-Schicht bildet und ein zusätzlicher Verlust an Lithium-Ionen (LLI) entsteht.\\
Untersuchungen zeigen, dass hohe Temperaturen sich primär kathodenseitig auswirken, während niedrige Temperaturen insbesondere die Anode beeinträchtigen.

\section{Zusammenfassung}

Die Alterungsprozesse von Lithium-Ionen-Batterien sind komplex und interdependent. Ein fundiertes Verständnis sowohl interner als auch externer Einflussfaktoren ist essenziell, um die relevanten Größen für die Modellierung auszuwählen. Die zuvor analysierten Mechanismen bilden die Grundlage für eine gezielte Datenvorverarbeitung, insbesondere im Hinblick auf reale Flottendaten aus Elektrofahrzeugen. Das \acs{BMS} liefert dabei kontinuierlich Messdaten, deren zeitlicher Verlauf entscheidend für die Identifikation der zugrunde liegenden Degradationspfade ist. Diese historischen Daten stellen die Basis für datengetriebene Methoden zur Vorhersage des Batteriezustands (\ac{SoH}) dar.\\
Das folgende Kapitel beschreibt den Stand der Technik im Bereich der \acs{SoH}-Prediktion mittels Machine Learning und neuronaler Netze, bevor in Kapitel~4 die verwendeten Flottendaten vorgestellt und aufbereitet werden.

\chapter{Anhang 2}
\label{apx:2}
