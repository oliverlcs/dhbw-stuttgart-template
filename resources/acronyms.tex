%!TEX root = ../main.tex

\addchap{Abkürzungsverzeichnis}

\begin{acronym} %[ABCDEF]               % längste Abkürzung
	\setlength{\itemsep}{-\parsep}     % Kein Abstand, kompakte Darstellung
	% Sortiere von Hand oder automatisch mit Kommandozeile (windows): sort file.txt /O file.txt

	\acro{DHBW}{Dualen Hochschule Baden-Württemberg}
	\acro{LIB}{Lithium-Ionen-Batterie}
	\acro{LFP}{Lithium-Eisenphosphat}
	\acro{NMC}{Nickel-Mangan-Cobalt}
	\acro{NCA}{Lithium-Nickel-Cobalt-Aluminium}
	\acro{Li+-Ionen}{Lithium-Ionen}
	\acro{LLI}{Loss of Lithium-Ions}
	\acro{LAM}{Loss of Active Material}
	\acro{LE}{Loss of Electrolyte}
	\acro{IR}{Internal Resistance}
	\acro{BMS}{Batterie Management System}
	\acro{SEI}{Solid Electrolyte Interface}
	\acro{CEI}{Cathode Electrolyte Interface}
	\acro{SoC}{State of Charge}
	\acro{SoH}{State of Health}
	\acro{DOD}{Depth of Discharge}
	\acro{KF}{Kalman Filter}
	\acro{EKF}{Extended Kalman Filter}
	\acro{UKF}{Unscented Kalman Filter}
	\acro{PF}{Particle Filter}
	\acro{ML}{Machine Learning}
	\acro{NN}{Neuronale Netze}
	\acro{BMS}{Battery Management System}
	\acro{SVM}{Support Vector Machine}
	\acro{RVM}{Relevance Vector Machine}
	\acro{RUL}{Remaining Useful Life}


\end{acronym}
