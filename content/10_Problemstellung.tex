%!TEX root = ../main.tex

\chapter{Einleitung}
Angesichts des Klimawandels sind umweltschonende Fortbewegungsmittel besonders wichtig. Insbesondere Elektrofahrzeuge spielen eine zentrale Rolle bei der Reduktion von CO2-Emissionen und einer nachhaltigeren Mobilität \cite{urlIdParisKlimaabkommen}. Dafür müssen Elektrofahrzeuge langlebig sein, da die Produktion sehr ressourcenaufwendig ist \cite{urlIdUmwelteinflussLithiumBatterien}. Um die Lebensdauer realistisch abzuschätzen, sind genaue Vorhersagen zur Batteriealterung notwendig. Mit ihnen kann eine nachhaltigere Batterieentwicklung umgesetzt und das Vertrauen in Elektrofahrzeuge gestärkt werden.\\
Die Lebensdauer einer Batterie hängt sowohl vom Batterietyp, etwa Li-Ion- oder Festkörperbatterien \cite{urlIdBatterieAlterungLithiumBatterien}, \cite{urlIdBatterieAlterungVerschiedenerSoCLithiumBatterien}, \cite{urlIdBatterieAlterungFestkörperBatterien}, \cite{urlIdBatterieAlterungVerschiedenerSoCLithiumBatterien}, als auch vom Nutzungsprofil, das durch Fahrverhalten und Klimabedingungen beeinflusst wird \cite{urlIdBatterieAlterungTemperatur}, ab. Zudem ist die Vorhersage der Batteriealterung bei Flottenfahrzeugen häufig durch verrauschte und unvollständige Datensätze erschwert.\\
Diese Arbeit widmet sich der datengetriebenen Vorhersage der Batteriealterung von elektrischen Kundenfahrzeugen. Dabei werden Machine-Learning- und Time-Series-Forecasting-Algorithmen analysiert und hinsichtlich Genauigkeit und Praxistauglichkeit validiert.

\pagebreak
\section{Problemstellung}
Trotz identischer Batterien beeinflussen das individuelle Fahr- \cite{urlIdBatterieAlterungVerschiedenerSoCLithiumBatterien} und Ladeverhalten \cite{urlIdBatterieAlterungVerschiedenerSoCLithiumBatterien} sowie Umweltbedingungen \cite{urlIdBatterieAlterungTemperatur} den Alterungsprozess erheblich, wodurch eine verlässliche Lebensdauerprognose über Jahre hinweg komplex und unsicher wird. Diese Unsicherheit erhöht sich durch unvollständige, verrauschte Flottendaten \cite{idEigeneFlottenDaten} und unbekanntes zukünftiges Fahrverhalten zusätzlich.\\
Physikalisch-chemische Modelle bieten zwar präzise Alterungssimulation im Labor \cite{urlIdPhysicalAgingMethodsForLIBs}, sind jedoch aufgrund aufwändiger Modellierung, Parametrierung und Validierung kaum auf Flotten skalierbar. Datengetriebene Ansätze, wie maschinelles Lernen und neuronale Netze \cite{urlIdDataAgingMethodsForLIBs}, sind skalierbar, jedoch anfällig für verrauschte Daten und Überanpassung.\\
Damit erweist sich die Prognose der Batteriealterung von einzelnen Fahrzeugen in der Fahrzeugflotte als komplex, da Modellauswahl und Datenlage ihr eigenes Maß an Unsicherheit mit sich bringen. Um die gesamten Flottendaten bei der Alterungsprognose mit einzubeziehen, beschäftigt sich diese Arbeit mit dem datengetriebenen Ansatz.

\pagebreak
\section{Ziel der Arbeit}
Das Ziel dieser Arbeit ist die Integretation eines bestehenden Algorithmus zur Prognose der Batteriealterung in Elektrofahrzeugen, basierend auf realen Fahrzeugflottendaten. Der Fokus liegt dabei auf der Zeitreihenvorhersage des Batterie-Gesundheitszustands über mehrere Jahre für einzelne Fahrzeuge. Präzise Prognosen tragen zur verbesserten Bewertung des Gesundheitszustands bei und ermöglichen eine nachhaltigere Nutzung der Batterien.
\\
Die Grundlage der Vorhersagen bilden die Fahrzeugflottendaten, die typischerweise Rauschen, Ungenauigkeiten und Messlücken aufweisen \cite{idEigeneFlottenDaten}. Mit synthetischen Daten können diese Probleme minimiert werden, jedoch sind die generierten Daten kaum validierbar. Daher verzichtet die Arbeit bewusst auf die Verwendug synthetischer Daten.
\\
Zur Zielerreichung werden Batterieaufbau sowie relevante Alterungsprozesse untersucht, um letztere mittels datengetriebener Modellen wie Machine Learning und neuronalen Netzen vorherzusagen. Die Datenaufbereitung vor dem Modelltraining umfasst die Bereinigung des bestehenden Datensatzes von Ausreißern und Messlücken sowie der Identifikation der wichtigsten Einflussfaktoren auf die Alterung. Beim Training des Algorithmus führen Überanpassung (Overfitting) und Modellwahl zu Unsicherheiten. Ersteres entsteht besonders bei kleinen oder stark korrelierten Datensätzen, sodass das Modell Details statt Muster erlernt und nicht robust gegenüber neuen Daten ist. Letzeres beschreibt, dass unterschiedliche Modelle (z.B. RNN, LSTM, GPR, XGBoost) unterschiedlich stark auf Datenmängel und verrauschten Daten reagieren. Die geeignete Wahl der Modelle wird in Kapitel 2 dargelegt.
Die Validierung der Modellergebnisse erfolgt anhand typischer Degradationsmuster von \ac{LIB} aus der Literatur, um realitätsnahe Ergebnisse zu liefern. Dabei grenzt sich die Arbeit bewussst von physikalsisch-chemischen Modellierungen und Laboruntersuchungen ab. (Wie kann ich stattdessen die Modellergebnisse validieren: Expertenwissen? Physikalisch Modelle als Referenz?)
\\
Der Erfolg des entwickelten Verfahrens wird abschließend anhand Fehlermaßen wie RMSE oder MAE sowie die Robustheit gegenüber verrauschten und unvollständigen Daten und die Fähigkeit zur Generalisierung auf unterschiedliche Fahrzeugprofile innerhalb der Flotte bewertet.
\pagebreak
\section{Vorgehensweise}
\paragraph{Disclaimer} In zukünftigen Version bindet das Kapitel Ziel der Arbeit die Strukturierung als Fließtext ein.
 
\begin{itemize}
    \item \textbf{Theoretische Grundlage}: Ausführung der theoretischen Grundlagen der Batteriealterung.
    \item \textbf{Modellanalyse}: Auswahl und Analyse geeigneter Machine-Learning- und Zeitreihen-Algorithmen.
    \item \textbf{Datenanalyse}: Untersuchung der Fahrzeugflottendatten hinsichtlich Qualität, Rauschen und Ausreißer.
    \item \textbf{Feature-Identifikation}: Identifikation relevanter Features für die Batteriealterung.
    \item \textbf{Datenvorverarbeitung}: Vorverarbeitung und Filtern der Fahrzeugflottendaten.
    \item \textbf{Modelltraining und -validierung}: Training und Validierung der Modelle anhand bekannten Alterungsverläufen.
    \item \textbf{Bewertung}: Vergleich der Modellergebnisse unter Berücksichtigung von Genauigkeit, Robustheit und Generalisierbarkeit.
\end{itemize}
