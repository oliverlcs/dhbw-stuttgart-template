%!TEX root = ../main.tex

\chapter{Implementierung und Ergebnisse}

\section{Implementierungsschritte}

- Preprocessing, Modellaufbau, Datenaufteilung, Training

\section{Evaluierung und Interpretation der Ergebnisse}

- Fehlermetriken, Plots, Validierung
- Interpretation, Bewertung und Kontextualisierung der Ergebnisse

\section{Limitationen}

- Datenbezogene Einschränkungen: Fehlende Werte, verrauschte Messwerte
- Modellgrenzen: Black-Box-Charakter von XGBoost oder NNs → schlechte Interpretierbarkeit
- Generalisierungsprobleme auf andere Flotten / Zellchemien
- Rechenzeit oder Ressourcenlimitierungen:
- Vereinfachungen: Fahrzeugnutzung innerhalb der Flotte bleibt konstant
- Keine Echtzeitfähigkeit: Falls dein Modell viele Daten braucht oder nicht online-adaptiv ist.

Der Text soll knapp und klar sein und die wesentlichen Gedanken der Arbeit beinhalten. Ein gewähltes Verfahren oder ein bestimmter Lösungsweg muss begründet werden. Es ist nicht notwendig, alle Vorversuche einzeln zu schildern. Bei Versuchen sind Voraussetzungen und Vernachlässigungen sowie die Anordnung, Leistungsfähigkeit und Messgenauigkeit der Versuchsanordnung anzugeben. \\ Die Ergebnisse der Arbeit sind unter Berücksichtigung der Voraussetzungen ausführlich zu diskutieren und mit den bereits bekannten Anschauungen und Erfahrungen zu vergleichen. \\ Ziel der Arbeit ist es, eindeutige Folgerungen und Richtlinien für die Praxis zu finden.