%!TEX root = ../main.tex

\chapter{Datengrundlage und Methodik}

\section{Systemübersicht}

- Datenflussdiagramm

\section{Analyse der Flottendaten und Feature-Auswahl}

Beschreibung:
- Woher kommen die Daten? (Qelle, Messsystem, Erhebungszeitraum, Direkt messende Daten wenn bestimmte Randbedingungen erfüllt sind -> ungenau und problematisch)
- Struktur der Daten? (Tabellarisch, Zeitreihen, Pro Fahrzeug oder aggregiert)
- Was enthalten sie? (Anzahl der Fahrzeuge, Zyklen, welche Messgrößen)
- Wie vollständig sind sie? (Fehlende Werte, Samplingrate)

Analyse:
- Korrelationsmatrix, ANOVA, Chi-Quadrat-Test
- Verteilung einzelner Merkmale
- Zeitliche Verläufe: Trends, Saisonalitäten
- Ausreißer und Anomielen: Extremwerte, unplausible Datenpunkte
- Missing Values: Wie viele, Muster erkennbar?
- Tools: Boxplots, Heatmaps, Scatterplots, line plots über Zeit, Pearson-Korrelation, PCA, Clustering
- Ziel: relevanten Features erkennen

Feature Auswahl:
- redundante, irrelevante oder korrelierte Features entfernen
Ziel: Vorhersagekraft maximieren und Overfitting vermeiden

\section{Auswahl der Machine Learning-Methodik}